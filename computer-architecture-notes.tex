\documentclass[12pt, a4paper]{article}
% \usepackage{ctex}

\usepackage[super, square]{natbib}
\usepackage[margin = 1in]{geometry}
\usepackage{
  color,
  clrscode,
  amssymb,
  ntheorem,
  amsmath,
  listings,
  fontspec,
  xcolor,
  supertabular,
  multirow
}
\definecolor{bgGray}{RGB}{36, 36, 36}
\usepackage[
  colorlinks,
  linkcolor=bgGray,
  anchorcolor=blue,
  citecolor=black
]{hyperref}
\newfontfamily\courier{Courier}

\theoremstyle{margin}
\theorembodyfont{\normalfont}

\newtheorem{theorem}{Theorem}
\newtheorem{definition}[theorem]{Definition}
\newtheorem{example}[theorem]{Example}

\newcommand{\st}{\text{s.t.}}
\newcommand{\mn}{\mathnormal}
\newcommand{\tbf}{\textbf}
\newcommand{\fl}{\mathnormal{fl}}
\newcommand{\f}{\mathnormal{f}}
\newcommand{\g}{\mathnormal{g}}
\newcommand{\R}{\mathbf{R}}
\newcommand{\Q}{\mathbf{Q}}
\newcommand{\JD}{\textbf{D}}
\newcommand{\rd}{\mathrm{d}}
\newcommand{\str}{^*}
\newcommand{\vep}{\varepsilon}
\newcommand{\lhs}{\text{L.H.S}}
\newcommand{\rhs}{\text{R.H.S}}
\newcommand{\con}{\text{Const}}
\newcommand{\oneton}{1,\,2,\,\dots,\,n}
\newcommand{\aoneton}{a_1a_2\dots a_n}
\newcommand{\xoneton}{x_1,\,x_2,\,\dots,\,x_n}


\title{Lecture Notes of Computer Architecture}
\author{Zhou Fan\\ACM Class, Shanghai Jiao Tong University}
\date{}

\begin{document}

\lstset{numbers=left,
  basicstyle=\scriptsize\courier,
  numberstyle=\tiny\courier\color{red!89!green!36!blue!36},
  language=C++,
  breaklines=true,
  keywordstyle=\color{blue!70},commentstyle=\color{red!50!green!50!blue!50},
  morekeywords={},
  stringstyle=\color{purple},
  frame=shadowbox,
  rulesepcolor=\color{red!20!green!20!blue!20}
}
\maketitle
\tableofcontents
\newpage

\section{Pipelining}
  \subsection{Introduction of Pipelining}
    \emph{Pipelining} is an implementation technique whereby multiple instructions are overlapped in execution; it takes advantage of parallelism that exists among the actions needed to execute an instruction.\cite{caqa}
    \subsubsection{Laundry Example}
      Suppose we have many loads of clothes to wash, dry and fold. 
      \begin{itemize}
        \item Each step (washing, drying and folding) is called a \emph{pipe stage} or a \emph{pipe segment}. 
        \item \emph{Latency} is the total time spent on single task, which is not improved by pipelining.Unbalanced lengths of pipe stages reduces speedup. 
        \item \emph{Throughput} is defined as the number of loads of clothes per minute. It shows how often a load of clothes exits the pipeline.
        \item The time required between moving an instruction one step down the pipeline is a \emph{processor cycle}. In a computer, this processor cycle is usually 1 clock cycle.
      \end{itemize}
    \subsubsection{Speedup from Pipelining}
      \begin{itemize}
        \item Unbalanced lengths of pipe stages reduces speedup. 
        \item Handover time between pipe stages reduces speedup.
      \end{itemize}
      To improve the efficiency of a pipeline, one should balance the length of each pipeline stage. If the stages are perfectly balanced, then the time per instruction on the pipeline processor is equal to (under ideal conditions) 
      $$\frac{\text{Time per instruction on unpipelined machine}}{\text{Number of pipe stages}}$$
      and the throughput of the pipeline is equal to
      $$\text{Number of pipe stages} \times \text{Throughput on unpipelined machine}$$


\begin{thebibliography}{9}
  \bibitem{caqa} 
    John L. Hennessy, David A. Patterson, et al.
    \emph{Computer Architecture: A Quantitative Approach},
    Fifth Edition, 2012. 
\end{thebibliography}

\end{document}